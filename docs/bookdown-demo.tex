% Options for packages loaded elsewhere
\PassOptionsToPackage{unicode}{hyperref}
\PassOptionsToPackage{hyphens}{url}
%
\documentclass[
]{book}
\usepackage{amsmath,amssymb}
\usepackage{lmodern}
\usepackage{ifxetex,ifluatex}
\ifnum 0\ifxetex 1\fi\ifluatex 1\fi=0 % if pdftex
  \usepackage[T1]{fontenc}
  \usepackage[utf8]{inputenc}
  \usepackage{textcomp} % provide euro and other symbols
\else % if luatex or xetex
  \usepackage{unicode-math}
  \defaultfontfeatures{Scale=MatchLowercase}
  \defaultfontfeatures[\rmfamily]{Ligatures=TeX,Scale=1}
\fi
% Use upquote if available, for straight quotes in verbatim environments
\IfFileExists{upquote.sty}{\usepackage{upquote}}{}
\IfFileExists{microtype.sty}{% use microtype if available
  \usepackage[]{microtype}
  \UseMicrotypeSet[protrusion]{basicmath} % disable protrusion for tt fonts
}{}
\makeatletter
\@ifundefined{KOMAClassName}{% if non-KOMA class
  \IfFileExists{parskip.sty}{%
    \usepackage{parskip}
  }{% else
    \setlength{\parindent}{0pt}
    \setlength{\parskip}{6pt plus 2pt minus 1pt}}
}{% if KOMA class
  \KOMAoptions{parskip=half}}
\makeatother
\usepackage{xcolor}
\IfFileExists{xurl.sty}{\usepackage{xurl}}{} % add URL line breaks if available
\IfFileExists{bookmark.sty}{\usepackage{bookmark}}{\usepackage{hyperref}}
\hypersetup{
  pdftitle={Cellular Tracking Technologies: Data Tools},
  pdfauthor={Jessica Gorzo},
  hidelinks,
  pdfcreator={LaTeX via pandoc}}
\urlstyle{same} % disable monospaced font for URLs
\usepackage{color}
\usepackage{fancyvrb}
\newcommand{\VerbBar}{|}
\newcommand{\VERB}{\Verb[commandchars=\\\{\}]}
\DefineVerbatimEnvironment{Highlighting}{Verbatim}{commandchars=\\\{\}}
% Add ',fontsize=\small' for more characters per line
\usepackage{framed}
\definecolor{shadecolor}{RGB}{248,248,248}
\newenvironment{Shaded}{\begin{snugshade}}{\end{snugshade}}
\newcommand{\AlertTok}[1]{\textcolor[rgb]{0.94,0.16,0.16}{#1}}
\newcommand{\AnnotationTok}[1]{\textcolor[rgb]{0.56,0.35,0.01}{\textbf{\textit{#1}}}}
\newcommand{\AttributeTok}[1]{\textcolor[rgb]{0.77,0.63,0.00}{#1}}
\newcommand{\BaseNTok}[1]{\textcolor[rgb]{0.00,0.00,0.81}{#1}}
\newcommand{\BuiltInTok}[1]{#1}
\newcommand{\CharTok}[1]{\textcolor[rgb]{0.31,0.60,0.02}{#1}}
\newcommand{\CommentTok}[1]{\textcolor[rgb]{0.56,0.35,0.01}{\textit{#1}}}
\newcommand{\CommentVarTok}[1]{\textcolor[rgb]{0.56,0.35,0.01}{\textbf{\textit{#1}}}}
\newcommand{\ConstantTok}[1]{\textcolor[rgb]{0.00,0.00,0.00}{#1}}
\newcommand{\ControlFlowTok}[1]{\textcolor[rgb]{0.13,0.29,0.53}{\textbf{#1}}}
\newcommand{\DataTypeTok}[1]{\textcolor[rgb]{0.13,0.29,0.53}{#1}}
\newcommand{\DecValTok}[1]{\textcolor[rgb]{0.00,0.00,0.81}{#1}}
\newcommand{\DocumentationTok}[1]{\textcolor[rgb]{0.56,0.35,0.01}{\textbf{\textit{#1}}}}
\newcommand{\ErrorTok}[1]{\textcolor[rgb]{0.64,0.00,0.00}{\textbf{#1}}}
\newcommand{\ExtensionTok}[1]{#1}
\newcommand{\FloatTok}[1]{\textcolor[rgb]{0.00,0.00,0.81}{#1}}
\newcommand{\FunctionTok}[1]{\textcolor[rgb]{0.00,0.00,0.00}{#1}}
\newcommand{\ImportTok}[1]{#1}
\newcommand{\InformationTok}[1]{\textcolor[rgb]{0.56,0.35,0.01}{\textbf{\textit{#1}}}}
\newcommand{\KeywordTok}[1]{\textcolor[rgb]{0.13,0.29,0.53}{\textbf{#1}}}
\newcommand{\NormalTok}[1]{#1}
\newcommand{\OperatorTok}[1]{\textcolor[rgb]{0.81,0.36,0.00}{\textbf{#1}}}
\newcommand{\OtherTok}[1]{\textcolor[rgb]{0.56,0.35,0.01}{#1}}
\newcommand{\PreprocessorTok}[1]{\textcolor[rgb]{0.56,0.35,0.01}{\textit{#1}}}
\newcommand{\RegionMarkerTok}[1]{#1}
\newcommand{\SpecialCharTok}[1]{\textcolor[rgb]{0.00,0.00,0.00}{#1}}
\newcommand{\SpecialStringTok}[1]{\textcolor[rgb]{0.31,0.60,0.02}{#1}}
\newcommand{\StringTok}[1]{\textcolor[rgb]{0.31,0.60,0.02}{#1}}
\newcommand{\VariableTok}[1]{\textcolor[rgb]{0.00,0.00,0.00}{#1}}
\newcommand{\VerbatimStringTok}[1]{\textcolor[rgb]{0.31,0.60,0.02}{#1}}
\newcommand{\WarningTok}[1]{\textcolor[rgb]{0.56,0.35,0.01}{\textbf{\textit{#1}}}}
\usepackage{longtable,booktabs,array}
\usepackage{calc} % for calculating minipage widths
% Correct order of tables after \paragraph or \subparagraph
\usepackage{etoolbox}
\makeatletter
\patchcmd\longtable{\par}{\if@noskipsec\mbox{}\fi\par}{}{}
\makeatother
% Allow footnotes in longtable head/foot
\IfFileExists{footnotehyper.sty}{\usepackage{footnotehyper}}{\usepackage{footnote}}
\makesavenoteenv{longtable}
\usepackage{graphicx}
\makeatletter
\def\maxwidth{\ifdim\Gin@nat@width>\linewidth\linewidth\else\Gin@nat@width\fi}
\def\maxheight{\ifdim\Gin@nat@height>\textheight\textheight\else\Gin@nat@height\fi}
\makeatother
% Scale images if necessary, so that they will not overflow the page
% margins by default, and it is still possible to overwrite the defaults
% using explicit options in \includegraphics[width, height, ...]{}
\setkeys{Gin}{width=\maxwidth,height=\maxheight,keepaspectratio}
% Set default figure placement to htbp
\makeatletter
\def\fps@figure{htbp}
\makeatother
\setlength{\emergencystretch}{3em} % prevent overfull lines
\providecommand{\tightlist}{%
  \setlength{\itemsep}{0pt}\setlength{\parskip}{0pt}}
\setcounter{secnumdepth}{5}
\usepackage{booktabs}
\usepackage{amsthm}
\makeatletter
\def\thm@space@setup{%
  \thm@preskip=8pt plus 2pt minus 4pt
  \thm@postskip=\thm@preskip
}
\makeatother
\ifluatex
  \usepackage{selnolig}  % disable illegal ligatures
\fi
\usepackage[]{natbib}
\bibliographystyle{apalike}

\title{Cellular Tracking Technologies: Data Tools}
\author{Jessica Gorzo}
\date{2021-08-16}

\begin{document}
\maketitle

{
\setcounter{tocdepth}{1}
\tableofcontents
}
\hypertarget{instructions}{%
\chapter*{Instructions}\label{instructions}}
\addcontentsline{toc}{chapter}{Instructions}

\includegraphics[width=4.17in]{bookdown-demo_files/figure-html/unnamed}

This is a manual for the R tools hosted at \href{https://github.com/cellular-tracking-technologies/data_tools}{our GitHub repository}. A RStudio tutorial is beyond the scope of this readme, but there are great resources to get you \href{https://www.earthdatascience.org/courses/earth-analytics/document-your-science/setup-r-rstudio/}{started with installing R and RStudio}.

\hypertarget{how-to-use-github}{%
\section*{How to use GitHub}\label{how-to-use-github}}
\addcontentsline{toc}{section}{How to use GitHub}

\hypertarget{getting-set-up}{%
\subsection*{Getting Set Up}\label{getting-set-up}}
\addcontentsline{toc}{subsection}{Getting Set Up}

\begin{enumerate}
\def\labelenumi{\arabic{enumi}.}
\tightlist
\item
  Create an \href{https://github.com}{account}.
\item
  Work through chapters 6-12 \href{https://happygitwithr.com/install-git.html}{here} if you need to install git, and connect it all with RStudio:
\item
  Choose your own adventure from here: do you want the working branch you created synced with the main repository, or do you want your main branch synced? Once you've decided, move onto the next step.
\end{enumerate}

\begin{itemize}
\tightlist
\item
  having your working branch synced makes sure you can easily pull the latest files into your work space, but to work around that you should make your own copies of files you alter to make sure files don't conflict when you pull updates\\
\item
  having your main branch synced is a bit more of a conventional structure, and means that changes pulled won't automatically propagate to your working branch. you could e.g.~pull changes to the main branch, and use that as a reference to see what changes you want to pull into your working copy, and resolve conflicts before merging\\
\end{itemize}

\begin{enumerate}
\def\labelenumi{\arabic{enumi}.}
\setcounter{enumi}{3}
\tightlist
\item
  Follow the instructions (at least through 5) \href{https://r-bio.github.io/intro-git-rstudio/}{here} under ``How to do this using RStudio and GitHub?''\\
\end{enumerate}

\begin{itemize}
\tightlist
\item
  you don't need to enter the back ticks in the shell\\
\item
  this example is a bit misleading because it doesn't include the .git, copy the link to the clipboard like before\\
\item
  RESTART RSTUDIO BEFORE MOVING ONTO STEP 6 IN THIS TUTORIAL\\
\end{itemize}

\begin{enumerate}
\def\labelenumi{\arabic{enumi}.}
\setcounter{enumi}{4}
\tightlist
\item
  If you want to pull updates from here to your copy, see \href{https://happygitwithr.com/upstream-changes.html\#pull-changes-from-upstream}{chapter 31}.
\end{enumerate}

\hypertarget{result}{%
\subsection*{Result}\label{result}}
\addcontentsline{toc}{subsection}{Result}

By following these instructions, you should now\ldots{}

\begin{itemize}
\tightlist
\item
  have a local copy of the repository\\
\item
  be working on your own branch\\
\item
  have an upstream connection to the main CTT repository
\end{itemize}

\hypertarget{intro}{%
\chapter{API}\label{intro}}

\hypertarget{install-postgres}{%
\section{Install Postgres}\label{install-postgres}}

If you choose to create a database out of your data (fair warning: in the future, the analysis tools will be based on this structure) you will need to install PostgreSQL on your machine.

\begin{enumerate}
\def\labelenumi{\arabic{enumi}.}
\tightlist
\item
  For simplicity, set your Postgres user name to be the same as your computer user name. Otherwise, you will need to pass it as an argument to the connection\\
\item
  Create a database in Postgres owned by that user name. You may have to set a password, and you may have to pass that password as an argument to the connection
\end{enumerate}

\hypertarget{your-token}{%
\section{Your Token}\label{your-token}}

Please request an API token through \href{https://celltracktech.com/support-radio-api/}{this form}. The token will appear on your account page when the request is fulfilled.

\hypertarget{r-script-api_run.r}{%
\section{R script: api\_run.R}\label{r-script-api_run.r}}

\begin{enumerate}
\def\labelenumi{\arabic{enumi}.}
\tightlist
\item
  As with the other R tools, I would suggest creating your own copy of ``api\_run.R'' within your local repository, and modify that file.
\item
  Assign the API token you found above to the ``my\_token'' variable (line 6)\\
\item
  Set your ``outpath'' variable to wherever your files will live. If you have already been manually downloading files, use that as your ``outpath.''
\end{enumerate}

\begin{itemize}
\tightlist
\item
  The script will search that directory, and will only download files you haven't already downloaded.
\item
  It will create a nested folder structure within that directory in the following order: project name, station(s), file types, files
\end{itemize}

\begin{enumerate}
\def\labelenumi{\arabic{enumi}.}
\setcounter{enumi}{3}
\tightlist
\item
  If you do not want to create a database\ldots{}
\end{enumerate}

\begin{itemize}
\tightlist
\item
  comment out lines 8, 13-14
\item
  remove the ``conn'' argument from the get\_my\_data() function (line 11)
\end{itemize}

\begin{enumerate}
\def\labelenumi{\arabic{enumi}.}
\setcounter{enumi}{4}
\tightlist
\item
  If you do want to create a database locally, set ``db\_name'' to the name of the Postgres database you created (line 7)
\end{enumerate}

\hypertarget{terminal}{%
\section{Terminal}\label{terminal}}

Run ``Rscript \textless path to your copy of api\_run.R\textgreater{}'' on the command line to run the script outside of RStudio (recommended)

\hypertarget{start-here-example-scripts}{%
\chapter{Start Here: Example Scripts}\label{start-here-example-scripts}}

\begin{itemize}
\tightlist
\item
  ``example.R'' shows you example implementations of the data management and node health functions (also read comments, functions that produce files are commented out)
\item
  ``locate\_example.R'' is a template script for running the location functions
\end{itemize}

I suggest making your own copy of these scripts, renaming them, and modifying them with your file path inputs.

\hypertarget{about-the-functions}{%
\chapter{About the Functions}\label{about-the-functions}}

There is a sub-folder within this repo named ``functions'' which is full of, well, scripts that contain functions! You'll notice they're often called (via source()) at the top of the example scripts. This loads in the custom functions that I have written to handle CTT data. Ultimately, these will be rolled into an R package.

\hypertarget{data-manager}{%
\section{Data Manager}\label{data-manager}}

\hypertarget{load_data}{%
\subsection{load\_data}\label{load_data}}

\textbf{Description}\\
Loads data

\textbf{Usage}

\begin{Shaded}
\begin{Highlighting}[]
\FunctionTok{load\_data}\NormalTok{(}\AttributeTok{directory\_name=}\ConstantTok{NULL}\NormalTok{, }\AttributeTok{starttime=}\ConstantTok{NULL}\NormalTok{, }\AttributeTok{endtime=}\ConstantTok{NULL}\NormalTok{, }\AttributeTok{tags=}\ConstantTok{NULL}\NormalTok{)  }
\end{Highlighting}
\end{Shaded}

\textbf{Arguments}

\begin{itemize}
\tightlist
\item
  directory\_name: the input folder can contain a miscellany of raw downloaded files from the sensor station (beep data, node health, GPS) all in the same folder or subfolders. Zipped folders need to be unzipped, but compressed files do not (i.e.~csv.gz files are just fine as they are).\\
\item
  starttime: start time in POSIXct\\
\item
  endtime: end time in POSIXct\\
\item
  tags: a vector of tag IDs
\end{itemize}

\textbf{Value}\\
The function will return a nested list where each item corresponds to:\\
1. beep data\\
2. node health\\
3. GPS

Within each list item, there is a list for a data frame and the hardware version. Also, a column ``v'' has been added to each data frame indicating the hardware version.

\hypertarget{node-health}{%
\section{Node Health}\label{node-health}}

\hypertarget{node_channel_plots}{%
\subsection{node\_channel\_plots}\label{node_channel_plots}}

\textbf{Description}\\
This function is the ``engine'' behind the export function. You can run it standalone with the following parameters, but you don't have to if your sole goal is to output image files.

\textbf{Usage}

\begin{Shaded}
\begin{Highlighting}[]
\FunctionTok{node\_channel\_plots}\NormalTok{(health, freq, ids, }\AttributeTok{lat=}\ConstantTok{NULL}\NormalTok{, }\AttributeTok{lon=}\ConstantTok{NULL}\NormalTok{)  }
\end{Highlighting}
\end{Shaded}

\textbf{Arguments}

\begin{itemize}
\tightlist
\item
  health: the 2nd list item output by the load\_data() function\\
\item
  freq: the time interval for which you want variables to be summarized\\
\item
  ids: a vector of IDs; the ID is of the format "\_"\\
\item
  lat: latitude (optional to produce day/night shading)
\item
  lon: longitude (optional to produce day/night shading)
\end{itemize}

\textbf{Value}\\
The output is a nested list, where the top level is each combination of channel and node, and each item is a list of the following plots:

\begin{enumerate}
\def\labelenumi{\arabic{enumi}.}
\tightlist
\item
  battery\\
\item
  RSSI\\
\item
  number of check-ins\\
\item
  scaled number of check-ins as line plot over scaled RSSI\\
\item
  box plot of node RSSI\\
  THE FOLLOWING ONLY FOR V2\\
\item
  latitude\\
\item
  longitude\\
\item
  scaled RSSI\\
\item
  dispersion
\end{enumerate}

\hypertarget{node_plots}{%
\subsection{node\_plots}\label{node_plots}}

\textbf{Description}\\
A set of diagnostic plots per node

\textbf{Usage}

\begin{Shaded}
\begin{Highlighting}[]
\FunctionTok{node\_plots}\NormalTok{(health, nodes, freq, }\AttributeTok{lat =} \ConstantTok{NULL}\NormalTok{, }\AttributeTok{lon =} \ConstantTok{NULL}\NormalTok{)}
\end{Highlighting}
\end{Shaded}

\textbf{Arguments}

\begin{itemize}
\tightlist
\item
  health: the 2nd data frame output by the load\_data() function\\
\item
  nodes: list of nodes\\
\item
  freq: the time interval for which you want variables to be summarized\\
\item
  lat: latitude\\
\item
  lon: longitude
\end{itemize}

\textbf{Value}\\
The output is a nested list for each node, with the following plots for each:

\begin{enumerate}
\def\labelenumi{\arabic{enumi}.}
\tightlist
\item
  RSSI\\
\item
  number of check-ins\\
\item
  battery\\
  THE FOLLOWING ONLY FOR V2\\
\item
  time mismatches\\
\item
  small time mismatches
\end{enumerate}

\hypertarget{gps_plots}{%
\subsection{gps\_plots}\label{gps_plots}}

\textbf{Description}\\
Plots to visualize some GPS data. ONLY FOR V2 HARDWARE

\textbf{Usage}

\begin{Shaded}
\begin{Highlighting}[]
\FunctionTok{gps\_plots}\NormalTok{(gps, freq)  }
\end{Highlighting}
\end{Shaded}

\textbf{Arguments}

\begin{itemize}
\tightlist
\item
  gps: the 3rd data frame from the load\_data() function\\
\item
  freq: the time interval of summary
\end{itemize}

\textbf{Value}\\
A list of the following plots:

\begin{enumerate}
\def\labelenumi{\arabic{enumi}.}
\tightlist
\item
  altitude\\
\item
  number of fixes
\end{enumerate}

\hypertarget{export_node_channel_plots}{%
\subsection{export\_node\_channel\_plots}\label{export_node_channel_plots}}

\textbf{Description}\\
Export plots of node x channel data

\textbf{Usage}

\begin{Shaded}
\begin{Highlighting}[]
\FunctionTok{export\_node\_channel\_plots}\NormalTok{(}\AttributeTok{plotlist=}\ConstantTok{NULL}\NormalTok{,health,}\AttributeTok{freq=}\StringTok{"1 hour"}\NormalTok{,}\AttributeTok{out\_path=}\FunctionTok{getwd}\NormalTok{(),}\AttributeTok{whichplots =} \FunctionTok{c}\NormalTok{(}\DecValTok{3}\NormalTok{,}\DecValTok{2}\NormalTok{,}\DecValTok{1}\NormalTok{))}
\end{Highlighting}
\end{Shaded}

\textbf{Arguments}

\begin{itemize}
\tightlist
\item
  plotlist: allows you to pass the output of node\_channel\_plots() if you prefer\\
\item
  health: the 2nd data frame output by the load\_data() function\\
\item
  freq: the time interval for which you want variables to be summarized\\
\item
  out\_path: where you want your plots to go\\
\item
  whichplots: an index vector of of the available plots
\end{itemize}

\textbf{Output}\\
This outputs a png for each input combination of node and channel.

\hypertarget{export_node_plots}{%
\subsection{export\_node\_plots}\label{export_node_plots}}

\textbf{Description}\\
Same as above; index for the plots can be chosen from the list under the \texttt{node\_plots()} description\\
\textbf{Usage}

\begin{Shaded}
\begin{Highlighting}[]
\FunctionTok{export\_node\_plots}\NormalTok{(}\AttributeTok{plotlist =} \ConstantTok{NULL}\NormalTok{, health,freq,}\AttributeTok{out\_path=}\FunctionTok{getwd}\NormalTok{(), }\AttributeTok{x=}\DecValTok{2}\NormalTok{, }\AttributeTok{y=}\DecValTok{3}\NormalTok{, }\AttributeTok{z=}\DecValTok{1}\NormalTok{)  }
\end{Highlighting}
\end{Shaded}

\textbf{Arguments}\\
To assign x, y and z, look at the description for \texttt{node\_channel\_plots()} and select those plot index in the order you want them on the page.

\textbf{Output}\\
This outputs a png for each input node

\hypertarget{calibration}{%
\chapter{Calibration}\label{calibration}}

\hypertarget{description}{%
\section{Description}\label{description}}

In order to best use the triangulation approach, calibration needs to be performed on your equipment. The goal of the calibration is to come up with a tag-specific relationship between RSSI and distance that is appropriate for your study site. An example of how to calibrate your system can be found in the \href{https://academic.oup.com/beheco/article/31/4/873/5840921?login=true\#supplementary-data}{supplementary material} of \citet{bircher2020extraterritorial}. Notice this calibration experiment demonstrates calibration of tags and receivers (in our case, nodes).

\hypertarget{tag-calibration}{%
\section{Tag Calibration}\label{tag-calibration}}

\begin{itemize}
\tightlist
\item
  transects to cover differently vegetated areas of the study site\\
\item
  varying tag height at each distance along the transect\\
\item
  simulations of movement and different orientations
\end{itemize}

\hypertarget{node-calibration}{%
\section{Node Calibration}\label{node-calibration}}

Tags at varying orientations a fixed distance from each node.

\hypertarget{how-to-use-the-output}{%
\section{How to Use the Output}\label{how-to-use-the-output}}

There are a few routes to go from here. To extend the application by \citet{bircher2020extraterritorial}, the latter could be used to adjust RSSI values per node. Alternatively, one could derive a simple RSSI \textasciitilde{} distance relationship for each tag using the data above to \href{https://iotandelectronics.wordpress.com/2016/10/07/how-to-calculate-distance-from-the-rssi-value-of-the-ble-beacon/}{measure N} and input custom relationships (see 5.2.2).

\hypertarget{localization-methods}{%
\chapter{Localization Methods}\label{localization-methods}}

\hypertarget{primitive-weighted-average}{%
\section{Primitive: Weighted Average}\label{primitive-weighted-average}}

\textbf{Description}\\
This is simply a weighted average based on number of beeps on a node and max. RSSI values.

\textbf{Usage}

\begin{Shaded}
\begin{Highlighting}[]
\FunctionTok{weighted\_average}\NormalTok{(freq, beeps, node, }\AttributeTok{node\_health=}\ConstantTok{NULL}\NormalTok{, }\AttributeTok{MAX\_NODES=}\DecValTok{0}\NormalTok{, }\AttributeTok{tag\_id=}\ConstantTok{NULL}\NormalTok{, }\AttributeTok{calibrate =} \ConstantTok{NULL}\NormalTok{, }\AttributeTok{keep\_cols =} \ConstantTok{NULL}\NormalTok{, }\AttributeTok{latlng =} \ConstantTok{TRUE}\NormalTok{, }\AttributeTok{minRSSI =} \DecValTok{0}\NormalTok{)}
\end{Highlighting}
\end{Shaded}

\textbf{Arguments}

\begin{itemize}
\tightlist
\item
  freq: this is the interval a localization should be summarized over, and is in the form of an interval (e.g.~\texttt{"3\ min"})\\
\item
  beep\_data: raw beep data frame (e.g.~\texttt{all\_data{[}{[}1{]}{]}{[}{[}1{]}{]}} from example.R)\\
\item
  node: read in node file\\
\item
  node\_health: node health data frame
\item
  MAX\_NODES: the max number of nodes that should contribute to a localization. default = 0 means all nodes
\item
  tag\_id: a vector of tags to calculate locations for
\item
  calibrate: the session ID if you want to calculate over the entire duration a tag was left at a point
\item
  keep\_cols: if there are valuable columns that shouldn't be dropped
\item
  latlng: BUGGY DO NOT USE YET
\item
  minRSSI: the minimum RSSI of data used for the location estimate
\end{itemize}

\textbf{Value}\\
A SpatialPointsDataFrame of estimated locations

\hypertarget{triangulation-functions}{%
\section{Triangulation Functions}\label{triangulation-functions}}

\hypertarget{calibration-1}{%
\subsection{Calibration}\label{calibration-1}}

\textbf{Usage}\\
NOTE: This function returns a single relationship for the entire input! The example code to date also demonstrates the output of 1 distance \textasciitilde{} RSSI relationship. In order to calibrate per tag, it will be necessary to modify your code to loop over (e.g.~apply) each tag ID, sub-setting your input data accordingly.

You can also use this function if you e.g.~left tags at a known location in your grid for a period of time. This function preps the beep data frame for input into the triangulation function, and also implements a calibration by fitting an asymptotic function for RSSI and distance.

The calibration data frame needs the following column names:

\begin{itemize}
\tightlist
\item
  pt: this can be any identifier for a given location used in the calibration\\
\item
  session\_id: a character row identifier
\item
  start: the beginning of the time interval when the tag was placed at the point, in POSIXct UTC\\
\item
  end: the end of the time interval when the tag was placed at the point, in POSIXct UTC\\
\item
  TagId: the tag ID left at the point\\
\item
  TagLat: latitude of the point\\
\item
  TagLng: longitude of the point
\end{itemize}

\begin{Shaded}
\begin{Highlighting}[]
\FunctionTok{calibrate}\NormalTok{(beep\_data, calibration, nodes, }\AttributeTok{calibrate =} \ConstantTok{TRUE}\NormalTok{, }\AttributeTok{freq =} \StringTok{"3 min"}\NormalTok{, }\AttributeTok{max\_nodes =} \DecValTok{0}\NormalTok{)}
\end{Highlighting}
\end{Shaded}

\textbf{Arguments}\\
The option \texttt{calibrate\ =\ TRUE} is the default, and means that summary stats will be calculated over the entire time interval for each calibration location. Otherwise, pass \texttt{calibrate\ =\ FALSE,\ freq\ =\ \textless{}interval\textgreater{}} for the time interval of interest.

\begin{itemize}
\tightlist
\item
  beep\_data: beep data frame\\
\item
  calibration: data frame described above\\
\item
  nodes: node file\\
\item
  calibrate: whether or not the entire time interval a tag was at a point should be used for the estimation
\item
  freq: alternatively, specify an interval for location estimation
\item
  max\_nodes: how many nodes should contribute? default = 0 means all nodes
\end{itemize}

\textbf{Value}\\
This function returns a list, the items of which are\ldots{}

\begin{enumerate}
\def\labelenumi{\arabic{enumi}.}
\tightlist
\item
  data frame to be input into the triangulation\\
\item
  a (see below)\\
\item
  S (see below)\\
\item
  K (see below)
\end{enumerate}

\hypertarget{custom-distance-function}{%
\subsection{Custom Distance Function}\label{custom-distance-function}}

\textbf{Description}\\
You can pass a custom distance function to the triangulation, in the form of a string, that represents the relationship between RSSI and distance for your system. The string that you pass is the right side of the formula, where the left side is distance. The string needs to contain \texttt{x} which refers to RSSI. An example of an asymptotic relationship can be generated by the following function and the output of the calibrate function:

\textbf{Usage}

\begin{Shaded}
\begin{Highlighting}[]
\FunctionTok{relate}\NormalTok{(a, S, K)}
\end{Highlighting}
\end{Shaded}

\textbf{Arguments}\\
These are fitted coefficients from an \texttt{SSasymp()} model relating distance to RSSI

\begin{itemize}
\tightlist
\item
  a = R0 e.g.~the 2nd item returned from the calibrate function\\
\item
  S = exp(lrc) e.g.~the 3rd item returned from the calibrate function\\
\item
  K = Aysm e.g.~the 4th item returned from the calibrate function
\end{itemize}

\textbf{Value}\\
Inspect that string if you would like to instead create your own (e.g.~for tag-wise calibration)

\hypertarget{data-prep}{%
\subsection{Data Prep}\label{data-prep}}

\textbf{Description}\\
This function prepares beep data to be input into the triangulation.

\textbf{Usage}

\begin{Shaded}
\begin{Highlighting}[]
\FunctionTok{loc\_prep}\NormalTok{(beep\_data, nodes, freq) }
\end{Highlighting}
\end{Shaded}

\textbf{Arguments}

\begin{itemize}
\tightlist
\item
  beep\_data: beep data frame\\
\item
  nodes: node file\\
\item
  freq: interval to calculate locations over
\end{itemize}

\textbf{Value}\\
A data frame that can be used as input to the \texttt{triangulate()} function

\hypertarget{triangulation}{%
\subsection{Triangulation}\label{triangulation}}

\textbf{Description}\\
This performs the triangulation with an input data frame and defined distance relationship. You could e.g.~fit a relationship between distance and RSSI based on your calibration work.

\textbf{Usage}

\begin{Shaded}
\begin{Highlighting}[]
\FunctionTok{triangulate}\NormalTok{(all\_data, }\AttributeTok{rssi =} \SpecialCharTok{{-}}\DecValTok{100}\NormalTok{, }\AttributeTok{node =} \DecValTok{3}\NormalTok{, }\AttributeTok{distance =}\NormalTok{ relation)}
\end{Highlighting}
\end{Shaded}

\textbf{Arguments}

\begin{itemize}
\tightlist
\item
  all\_data: a formatted data frame, such as the output from \texttt{loc\_prep()} or the data frame returned by \texttt{calibrate()}\\
\item
  rssi: the minimum RSSI threshold to incorporate data into the location calculation\\
\item
  node: the maximum number of nodes to contribute to the calculation\\
\item
  distance: a string representing the right side of a formula relating RSSI to distance, where distance is the left side and RSSI is \texttt{x} in the string
\end{itemize}

\textbf{Value}\\
A data frame with estimated locations and error

  \bibliography{doc.bib}

\end{document}
