% Options for packages loaded elsewhere
\PassOptionsToPackage{unicode}{hyperref}
\PassOptionsToPackage{hyphens}{url}
%
\documentclass[
]{book}
\usepackage{lmodern}
\usepackage{amssymb,amsmath}
\usepackage{ifxetex,ifluatex}
\ifnum 0\ifxetex 1\fi\ifluatex 1\fi=0 % if pdftex
  \usepackage[T1]{fontenc}
  \usepackage[utf8]{inputenc}
  \usepackage{textcomp} % provide euro and other symbols
\else % if luatex or xetex
  \usepackage{unicode-math}
  \defaultfontfeatures{Scale=MatchLowercase}
  \defaultfontfeatures[\rmfamily]{Ligatures=TeX,Scale=1}
\fi
% Use upquote if available, for straight quotes in verbatim environments
\IfFileExists{upquote.sty}{\usepackage{upquote}}{}
\IfFileExists{microtype.sty}{% use microtype if available
  \usepackage[]{microtype}
  \UseMicrotypeSet[protrusion]{basicmath} % disable protrusion for tt fonts
}{}
\makeatletter
\@ifundefined{KOMAClassName}{% if non-KOMA class
  \IfFileExists{parskip.sty}{%
    \usepackage{parskip}
  }{% else
    \setlength{\parindent}{0pt}
    \setlength{\parskip}{6pt plus 2pt minus 1pt}}
}{% if KOMA class
  \KOMAoptions{parskip=half}}
\makeatother
\usepackage{xcolor}
\IfFileExists{xurl.sty}{\usepackage{xurl}}{} % add URL line breaks if available
\IfFileExists{bookmark.sty}{\usepackage{bookmark}}{\usepackage{hyperref}}
\hypersetup{
  pdftitle={Cellular Tracking Technologies: Data Tools},
  pdfauthor={Jessica Gorzo},
  hidelinks,
  pdfcreator={LaTeX via pandoc}}
\urlstyle{same} % disable monospaced font for URLs
\usepackage{color}
\usepackage{fancyvrb}
\newcommand{\VerbBar}{|}
\newcommand{\VERB}{\Verb[commandchars=\\\{\}]}
\DefineVerbatimEnvironment{Highlighting}{Verbatim}{commandchars=\\\{\}}
% Add ',fontsize=\small' for more characters per line
\usepackage{framed}
\definecolor{shadecolor}{RGB}{248,248,248}
\newenvironment{Shaded}{\begin{snugshade}}{\end{snugshade}}
\newcommand{\AlertTok}[1]{\textcolor[rgb]{0.94,0.16,0.16}{#1}}
\newcommand{\AnnotationTok}[1]{\textcolor[rgb]{0.56,0.35,0.01}{\textbf{\textit{#1}}}}
\newcommand{\AttributeTok}[1]{\textcolor[rgb]{0.77,0.63,0.00}{#1}}
\newcommand{\BaseNTok}[1]{\textcolor[rgb]{0.00,0.00,0.81}{#1}}
\newcommand{\BuiltInTok}[1]{#1}
\newcommand{\CharTok}[1]{\textcolor[rgb]{0.31,0.60,0.02}{#1}}
\newcommand{\CommentTok}[1]{\textcolor[rgb]{0.56,0.35,0.01}{\textit{#1}}}
\newcommand{\CommentVarTok}[1]{\textcolor[rgb]{0.56,0.35,0.01}{\textbf{\textit{#1}}}}
\newcommand{\ConstantTok}[1]{\textcolor[rgb]{0.00,0.00,0.00}{#1}}
\newcommand{\ControlFlowTok}[1]{\textcolor[rgb]{0.13,0.29,0.53}{\textbf{#1}}}
\newcommand{\DataTypeTok}[1]{\textcolor[rgb]{0.13,0.29,0.53}{#1}}
\newcommand{\DecValTok}[1]{\textcolor[rgb]{0.00,0.00,0.81}{#1}}
\newcommand{\DocumentationTok}[1]{\textcolor[rgb]{0.56,0.35,0.01}{\textbf{\textit{#1}}}}
\newcommand{\ErrorTok}[1]{\textcolor[rgb]{0.64,0.00,0.00}{\textbf{#1}}}
\newcommand{\ExtensionTok}[1]{#1}
\newcommand{\FloatTok}[1]{\textcolor[rgb]{0.00,0.00,0.81}{#1}}
\newcommand{\FunctionTok}[1]{\textcolor[rgb]{0.00,0.00,0.00}{#1}}
\newcommand{\ImportTok}[1]{#1}
\newcommand{\InformationTok}[1]{\textcolor[rgb]{0.56,0.35,0.01}{\textbf{\textit{#1}}}}
\newcommand{\KeywordTok}[1]{\textcolor[rgb]{0.13,0.29,0.53}{\textbf{#1}}}
\newcommand{\NormalTok}[1]{#1}
\newcommand{\OperatorTok}[1]{\textcolor[rgb]{0.81,0.36,0.00}{\textbf{#1}}}
\newcommand{\OtherTok}[1]{\textcolor[rgb]{0.56,0.35,0.01}{#1}}
\newcommand{\PreprocessorTok}[1]{\textcolor[rgb]{0.56,0.35,0.01}{\textit{#1}}}
\newcommand{\RegionMarkerTok}[1]{#1}
\newcommand{\SpecialCharTok}[1]{\textcolor[rgb]{0.00,0.00,0.00}{#1}}
\newcommand{\SpecialStringTok}[1]{\textcolor[rgb]{0.31,0.60,0.02}{#1}}
\newcommand{\StringTok}[1]{\textcolor[rgb]{0.31,0.60,0.02}{#1}}
\newcommand{\VariableTok}[1]{\textcolor[rgb]{0.00,0.00,0.00}{#1}}
\newcommand{\VerbatimStringTok}[1]{\textcolor[rgb]{0.31,0.60,0.02}{#1}}
\newcommand{\WarningTok}[1]{\textcolor[rgb]{0.56,0.35,0.01}{\textbf{\textit{#1}}}}
\usepackage{longtable,booktabs}
% Correct order of tables after \paragraph or \subparagraph
\usepackage{etoolbox}
\makeatletter
\patchcmd\longtable{\par}{\if@noskipsec\mbox{}\fi\par}{}{}
\makeatother
% Allow footnotes in longtable head/foot
\IfFileExists{footnotehyper.sty}{\usepackage{footnotehyper}}{\usepackage{footnote}}
\makesavenoteenv{longtable}
\usepackage{graphicx,grffile}
\makeatletter
\def\maxwidth{\ifdim\Gin@nat@width>\linewidth\linewidth\else\Gin@nat@width\fi}
\def\maxheight{\ifdim\Gin@nat@height>\textheight\textheight\else\Gin@nat@height\fi}
\makeatother
% Scale images if necessary, so that they will not overflow the page
% margins by default, and it is still possible to overwrite the defaults
% using explicit options in \includegraphics[width, height, ...]{}
\setkeys{Gin}{width=\maxwidth,height=\maxheight,keepaspectratio}
% Set default figure placement to htbp
\makeatletter
\def\fps@figure{htbp}
\makeatother
\setlength{\emergencystretch}{3em} % prevent overfull lines
\providecommand{\tightlist}{%
  \setlength{\itemsep}{0pt}\setlength{\parskip}{0pt}}
\setcounter{secnumdepth}{5}
\usepackage{booktabs}
\usepackage{amsthm}
\makeatletter
\def\thm@space@setup{%
  \thm@preskip=8pt plus 2pt minus 4pt
  \thm@postskip=\thm@preskip
}
\makeatother
\usepackage[]{natbib}
\bibliographystyle{apalike}

\title{Cellular Tracking Technologies: Data Tools}
\author{Jessica Gorzo}
\date{2021-05-17}

\begin{document}
\maketitle

{
\setcounter{tocdepth}{1}
\tableofcontents
}
\hypertarget{instructions}{%
\chapter*{Instructions}\label{instructions}}
\addcontentsline{toc}{chapter}{Instructions}

\includegraphics{bookdown-demo_files/figure-html/unnamed.png}

This is a manual for the R tools hosted at \href{https://github.com/cellular-tracking-technologies/data_tools}{our GitHub repository}. A RStudio tutorial is beyond the scope of this readme, but there are great resources to get you \href{https://www.earthdatascience.org/courses/earth-analytics/document-your-science/setup-r-rstudio/}{started with installing R and RStudio}.

\hypertarget{how-to-use-github}{%
\section*{How to use GitHub}\label{how-to-use-github}}
\addcontentsline{toc}{section}{How to use GitHub}

\begin{enumerate}
\def\labelenumi{\arabic{enumi}.}
\tightlist
\item
  Create an \href{https://github.com}{account}.
\item
  Work through chapters 6-12 \href{https://happygitwithr.com/install-git.html}{here} if you need to install git, and connect it all with RStudio:
\item
  Follow the instructions (at least through 5) \href{https://r-bio.github.io/intro-git-rstudio/}{here} under ``How to do this using RStudio and GitHub?''
\end{enumerate}

\begin{itemize}
\tightlist
\item
  you don't need to enter the backticks in the shell
\item
  this example is a bit misleading because it doesn't include the .git, copy the link to the clipboard like before
\item
  RESTART RSTUDIO BEFORE MOVING ONTO STEP 6 IN THIS TUTORIAL
\end{itemize}

\begin{enumerate}
\def\labelenumi{\arabic{enumi}.}
\setcounter{enumi}{3}
\tightlist
\item
  If you want to pull updates from here to your copy, see \href{https://happygitwithr.com/upstream-changes.html\#pull-changes-from-upstream}{chapter 31}.
\end{enumerate}

\hypertarget{intro}{%
\chapter{API}\label{intro}}

\hypertarget{install-postgres}{%
\section{Install Postgres}\label{install-postgres}}

If you choose to create a database out of your data (fair warning: in the future, the analysis tools will be based on this structure) you will need to install PostgreSQL on your machine.\\
1. For simplicity, set your Postgres user name to be the same as your computer user name. Otherwise, you will need to pass it as an argument to the connection\\
2. Create a database in Postgres owned by that user name. You may have to set a password, and you may have to pass that password as an argument to the connection

\hypertarget{your-token}{%
\section{Your Token}\label{your-token}}

Please request an API token through \href{https://celltracktech.com/support-radio-api/}{this form}. The token will appear on your account page when the request is fulfilled.

\hypertarget{r-script-api_run.r}{%
\section{R script: api\_run.R}\label{r-script-api_run.r}}

\begin{enumerate}
\def\labelenumi{\arabic{enumi}.}
\tightlist
\item
  As with the other R tools, I would suggest creating your own copy of ``api\_run.R'' within your local repository, and modify that file.
\item
  Assign the API token you found above to the ``my\_token'' variable (line 6)\\
\item
  Set your ``outpath'' variable to wherever your files will live. If you have already been manually downloading files, use that as your ``outpath.''
\end{enumerate}

\begin{itemize}
\tightlist
\item
  The script will search that directory, and will only download files you haven't already downloaded.
\item
  It will create a nested folder structure within that directory in the following order: project name, station(s), file types, files
\end{itemize}

\begin{enumerate}
\def\labelenumi{\arabic{enumi}.}
\setcounter{enumi}{3}
\tightlist
\item
  If you do not want to create a database\ldots{}
\end{enumerate}

\begin{itemize}
\tightlist
\item
  comment out lines 8, 13-14
\item
  remove the ``conn'' argument from the get\_my\_data() function (line 11)
\end{itemize}

\begin{enumerate}
\def\labelenumi{\arabic{enumi}.}
\setcounter{enumi}{4}
\tightlist
\item
  If you do want to create a database locally, set ``db\_name'' to the name of the Postgres database you created (line 7)
\end{enumerate}

\hypertarget{terminal}{%
\section{Terminal}\label{terminal}}

Run ``Rscript \textless path to your copy of api\_run.R\textgreater{}'' on the command line to run the script outside of RStudio (recommended)

\hypertarget{start-here-example-scripts}{%
\chapter{Start Here: Example Scripts}\label{start-here-example-scripts}}

\begin{itemize}
\tightlist
\item
  ``example.R'' shows you example implementations of the data management and node health functions (also read comments, functions that produce files are commented out)
\item
  ``locate\_example.R'' is a template script for running the location functions
\end{itemize}

I suggest making your own copy of these scripts, renaming them, and modifying them with your file path inputs.

\hypertarget{about-the-functions}{%
\chapter{About the Functions}\label{about-the-functions}}

There is a sub-folder within this repo named ``functions'' which is full of, well, scripts that contain functions! You'll notice they're often called (via source()) at the top of the example scripts. This loads in the custom functions that I have written to handle CTT data. Ultimately, these will be rolled into an R package.

\hypertarget{data-manager}{%
\section{Data Manager}\label{data-manager}}

\hypertarget{load_datainfile}{%
\subsection{load\_data(infile)}\label{load_datainfile}}

\textbf{Input}\\
The input folder (``infile'') can contain any melange of raw downloaded files from the sensor station (beep data, node health, GPS) all in the same folder or subfolders. Zipped folders need to be unzipped, but compressed files do not (i.e.~csv.gz files are just fine as they are).

\textbf{Output}\\
The function will return a list of 3 dataframes from the files in the folder you give it:\\
1. beep data\\
2. node health\\
3. GPS

\hypertarget{node-health}{%
\section{Node Health}\label{node-health}}

\hypertarget{node_channel_plotshealth-freq}{%
\subsection{node\_channel\_plots(health, freq)}\label{node_channel_plotshealth-freq}}

This function is the ``engine'' behind the export function. You can run it standalone with the following parameters, but you don't have to.

\textbf{Input}\\
1. health: the 2nd dataframe output by the load\_data() function\\
2. freq: the time interval for which you want variables to be summarized

\textbf{Output}
The output is a nested list for each combination of channel and node, with the following plots for each:

\begin{enumerate}
\def\labelenumi{\arabic{enumi}.}
\tightlist
\item
  battery\\
\item
  RSSI\\
\item
  number of check-ins\\
\item
  scaled number of check-ins as line plot over scaled RSSI\\
\item
  box plot of node RSSI
\end{enumerate}

\hypertarget{v2_plotshealth-freq}{%
\subsection{v2\_plots(health, freq)}\label{v2_plotshealth-freq}}

\textbf{Input}\\
1. health: the 2nd dataframe output by the load\_data() function\\
2. freq: the time interval for which you want variables to be summarized\\
\textbf{Output} The output is a nested list for each combination of channel and node, with the following plots for each:

\begin{enumerate}
\def\labelenumi{\arabic{enumi}.}
\tightlist
\item
  latitude\\
\item
  longitude\\
\item
  RSSI\\
\item
  dispersion
\end{enumerate}

\hypertarget{node_plotshealth-nodes-freq}{%
\subsection{node\_plots(health, nodes, freq)}\label{node_plotshealth-nodes-freq}}

NOTE: THIS ONLY WORKS FOR V2\\
\textbf{Input}\\
1. health: the 2nd dataframe output by the load\_data() function\\
2. nodes: list of nodes\\
3. freq: the time interval for which you want variables to be summarized\\
\textbf{Output} The output is a nested list for each node, with the following plots for each:

\begin{enumerate}
\def\labelenumi{\arabic{enumi}.}
\tightlist
\item
  RSSI\\
\item
  number of check-ins\\
\item
  battery\\
\item
  time mismatches\\
\item
  small time mismatches
\end{enumerate}

\hypertarget{gps_plotsgps-freq}{%
\subsection{gps\_plots(gps, freq)}\label{gps_plotsgps-freq}}

\textbf{Input}\\
1. gps: the 3rd data frame from the load\_data() function\\
2. freq: the time interval of summary\\
\textbf{Output}\\
1. altitude\\
2. number of fixes

\hypertarget{export_node_channel_plotshealth_data-freq-out_path-x-y-z}{%
\subsection{export\_node\_channel\_plots(health\_data, freq, out\_path, x, y, z)}\label{export_node_channel_plotshealth_data-freq-out_path-x-y-z}}

\textbf{Input}\\
1. health\_data: the 2nd dataframe output by the load\_data() function\\
2. freq: the time interval for which you want variables to be summarized\\
3. out\_path: where you want your plots to go\\
4. x: the plot for the 1st panel\\
5. y: the plot for the 2nd panel\\
6. z: the plot for the 3rd panel

To assign x, y and z, look at the description for node\_channel\_plots() and select those plot indices in the order you want them on the page.

\hypertarget{export_node_plotshealth_data-freq-out_path-x-y-z}{%
\subsection{export\_node\_plots(health\_data, freq, out\_path, x, y, z)}\label{export_node_plotshealth_data-freq-out_path-x-y-z}}

NOTE: THIS ONLY WORKS FOR V2\\
same as above; indices for the plots can be chosen from the list under the node\_plots() description

\hypertarget{localization-methods}{%
\chapter{Localization Methods}\label{localization-methods}}

\hypertarget{primitive-weighted-average}{%
\section{Primitive: Weighted Average}\label{primitive-weighted-average}}

This is simply a weighted average based on number of beeps on a node and max. RSSI values. It takes the following mandatory arguments:\\
* freq: this is the interval a localization should be summarized over, and is in the form of an interval (e.g.~``3 min'')\\
* beep\_data: raw beep data frame (e.g.~all\_data{[}{[}1{]}{]}{[}{[}1{]}{]} from example.R)\\
* node: read in node file

locations \textless- weighted\_average(freq = ``3 min'', beep\_data, node = nodes, node\_health, 0, tags)

\hypertarget{triangulation-functions}{%
\section{Triangulation Functions}\label{triangulation-functions}}

\hypertarget{data-prep}{%
\subsection{Data Prep}\label{data-prep}}

This function prepares beep data to be input into the triangulation.

\begin{Shaded}
\begin{Highlighting}[]
\NormalTok{input <-}\StringTok{ }\KeywordTok{loc_prep}\NormalTok{(beep_data, nodes, freq)  }
\end{Highlighting}
\end{Shaded}

\hypertarget{calibration}{%
\subsection{Calibration}\label{calibration}}

This function preps the beep data frame for input into the triangulation function, and also implements a calibration by fitting an asymptotic function for RSSI and distance.

The calibration data frame needs the following column names:\\
* pt: this can be any identifier for a given location used in the calibration\\
* session\_id: this can be any identifier for the time period the tag was at this spot\\
* start: the beginning of the time interval when the tag was placed at the point, in POSIXct UTC\\
* end: the end of the time interval when the tag was placed at the point, in POSIXct UTC\\
* TagId: the tag ID left at the point\\
* TagLat: latitude of the point\\
* TagLng: longitude of the point

The option \texttt{calibrate\ =\ TRUE} is the default, and means that summary stats will be calculated over the entire time interval for each calibration location. Otherwise, pass \texttt{calibrate\ =\ FALSE,\ freq\ =\ \textless{}interval\textgreater{}} for the time interval of interest.

\begin{Shaded}
\begin{Highlighting}[]
\NormalTok{calibrated <-}\StringTok{ }\KeywordTok{calibrate}\NormalTok{(beep_data, calibration, nodes, }\DataTypeTok{calibrate =} \OtherTok{TRUE}\NormalTok{)}
\end{Highlighting}
\end{Shaded}

This function returns a list, the items of which are\ldots{}\\
1. data frame to be input into the triangulation\\
2. a (see below)\\
3. S (see below)\\
4. K (see below)

\hypertarget{custom-distance-function}{%
\subsection{Custom Distance Function}\label{custom-distance-function}}

You can pass a custom distance function, in the form of a string, that represents the relationship between RSSI and distance for your system. The string is the right side of the formula, where the left side is distance. The string needs to contain \texttt{x} which refers to RSSI. An example of an asymptotic relationship can be generated by the following function and the output of the calibrate function:

\begin{Shaded}
\begin{Highlighting}[]
\NormalTok{relation <-}\StringTok{ }\KeywordTok{relate}\NormalTok{(calibrated[[}\DecValTok{2}\NormalTok{]], calibrated[[}\DecValTok{3}\NormalTok{]], calibrated[[}\DecValTok{4}\NormalTok{]])}
\end{Highlighting}
\end{Shaded}

\hypertarget{triangulation}{%
\subsection{Triangulation}\label{triangulation}}

This performs the triangulation with an input dataframe and defined distance relationship.

\begin{Shaded}
\begin{Highlighting}[]
\NormalTok{out <-}\StringTok{ }\KeywordTok{triangulate}\NormalTok{(calibrated[[}\DecValTok{1}\NormalTok{]], }\DataTypeTok{distance =}\NormalTok{ relation)}
\end{Highlighting}
\end{Shaded}

\end{document}
